\title{Guide to using PintOS}
\author{
		Prof: Jorge Gonzalez\\
			jgonzalez@utec.edu.pe\\
        TA: Martin Carrasco \\
         	martin.carrasco@utec.edu.pe
}
\date{\today}
\documentclass[12pt]{article}
\usepackage{hyperref}
\begin{document}

\maketitle

\section{Making a docker container for PintOS}

\begin{enumerate}
	\item Pull the docker image from Docker Hub
	\item Create a volume to have data be persistant e.g. \textit{sudo docker volume create my\_volume PATH} (see additional info in \texttt{item 5}).
	\item Create the Dockerfile \textit{(Use the template Dockerfile included to guide you, notice that in our template we already pull the image).}
	\begin{itemize}
		\item Install dependencies 
		\item Set enviromental variables 
	\end{itemize}
	\item Build the container e.g. \textit{sudo docker build -t pintos .}
	\item Run the container e.g. \textit{sudo docker run -it --volume my\_volume:/app --name pint-sim pintos}. For more details about using volumes, please refer to \href{https://docs.docker.com/storage/volumes/}{Docker Documentation}.
	
\end{enumerate}

\section{Seting up and compiling PintOS}
\begin{enumerate}
	\item Make sure all the dependencies were installed correctly
	\item Compile the following submodules
	\begin{itemize}
		\item \texttt{userprog}
		\item \texttt{vm}
		\item \texttt{filesys}
	\end{itemize}
	\item Edit \textbf{src/utils/Makefile} to replace \texttt{LDFLAGS= -lm} to \texttt{LDLIBS = -lm} then compile \textbf{src/utils}
	\item Edit  \textbf{src/thread/Make.vars} and change \texttt{SIMULATOR=} to \texttt{SIMULATOR= -qemu} then compile \textbf{src/threads}
	\item Change \textbf{src/utils/pintos} \texttt{\$sim=bochos} to \texttt{\$sim=qemu}
	\item Change \textbf{src/utils/pintos}, check \texttt{\$name = find\_file('kernel.bin')} to point to \textbf{threads/build/kernel.bin}
	\item  Change \textbf{src/utils/pintos} \texttt{my (@cmd) = ('qemu')} to \texttt{my (@cmd) =qemu-system-x86\_64}
	\item Edit in \textbf{src/utils/Pintos.pm}, \texttt{\$name = find\_file('loader.bin')} and point it to \textbf{threads/build/loader.bin}
	\item See \href{https://web.stanford.edu/class/cs140/projects/pintos/pintos_1.html#SEC4}{PintOS Documentation} for compile and run. Notice that we are using Qemu.
\end{enumerate}
\end{document}



